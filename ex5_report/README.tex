%%%%%%%%%%%%%%%%%%%%%%%%%%%%%%%%%%%%%%%%%
% Short Sectioned Assignment
% LaTeX Template
% Version 1.0 (5/5/12)
%
% This template has been downloaded from:
% http://www.LaTeXTemplates.com
%
% Original author:
% Frits Wenneker (http://www.howtotex.com)
%
% License:
% CC BY-NC-SA 3.0 (http://creativecommons.org/licenses/by-nc-sa/3.0/)
%
%%%%%%%%%%%%%%%%%%%%%%%%%%%%%%%%%%%%%%%%%

%----------------------------------------------------------------------------------------
%	PACKAGES AND OTHER DOCUMENT CONFIGURATIONS
%----------------------------------------------------------------------------------------

\documentclass[paper=a4, fontsize=11pt]{scrartcl} % A4 paper and 11pt font size

\usepackage[T1]{fontenc} % Use 8-bit encoding that has 256 glyphs
\usepackage{fourier} % Use the Adobe Utopia font for the document - comment this line to return to the LaTeX default
\usepackage[english]{babel} % English language/hyphenation
\usepackage{amsmath,amsfonts,amsthm} % Math packages

\usepackage{lipsum} % Used for inserting dummy 'Lorem ipsum' text into the template

\usepackage{sectsty} % Allows customizing section commands
\allsectionsfont{\centering \normalfont\scshape} % Make all sections centered, the default font and small caps

\usepackage{fancyhdr} % Custom headers and footers

% my packages
\usepackage{commath}
\usepackage{mathtools}
\usepackage{graphicx}
\usepackage{algorithm}
\usepackage[]{algpseudocode}
\DeclarePairedDelimiter{\ceil}{\lceil}{\rceil}
\usepackage{pgfplots}
\pgfplotsset{compat=newest}
\usepackage{hyperref}
\usepackage{enumitem}
\usepackage{subcaption}
\usepackage{multirow}
\usepackage{tkz-graph}
\usepackage{adjustbox}
\usepackage{cancel}

\newlist{filedescription}{description}{2}
\setlist[filedescription]{font=\normalfont\normalcolor\bfseries\itshape}

\newlist{paramdescription}{description}{1}
\setlist[paramdescription]{font=\normalfont\normalcolor\itshape}

\pagestyle{fancyplain} % Makes all pages in the document conform to the custom headers and footers
\fancyhead{} % No page header - if you want one, create it in the same way as the footers below
\fancyfoot[L]{} % Empty left footer
\fancyfoot[C]{} % Empty center footer
\fancyfoot[R]{\thepage} % Page numbering for right footer
\renewcommand{\headrulewidth}{0pt} % Remove header underlines
\renewcommand{\footrulewidth}{0pt} % Remove footer underlines
\setlength{\headheight}{13.6pt} % Customize the height of the header

\numberwithin{equation}{section} % Number equations within sections (i.e. 1.1, 1.2, 2.1, 2.2 instead of 1, 2, 3, 4)
\numberwithin{figure}{section} % Number figures within sections (i.e. 1.1, 1.2, 2.1, 2.2 instead of 1, 2, 3, 4)
\numberwithin{table}{section} % Number tables within sections (i.e. 1.1, 1.2, 2.1, 2.2 instead of 1, 2, 3, 4)

\setlength\parindent{0pt} % Removes all indentation from paragraphs - comment this line for an assignment with lots of text

% new commands
\newcommand{\filename}[1]{\textbf{\textit{#1}}}
\newcommand{\funcname}[1]{\textbf{#1}}
\newcommand{\inv}{^{\raisebox{.2ex}{$\scriptscriptstyle-1$}}}
\renewcommand{\vec}[1]{\mathbf{#1}}

\makeatletter
\renewcommand*\env@matrix[1][*\c@MaxMatrixCols c]{%
  \hskip -\arraycolsep
  \let\@ifnextchar\new@ifnextchar
  \array{#1}}
\makeatother

\makeatletter
\def\BState{\State\hskip-\ALG@thistlm}
\makeatother

\DeclareMathAlphabet{\mathcal}{OMS}{cmsy}{m}{n}
\DeclareMathOperator*{\argmin}{arg\,min} % Jan Hlavacek

%----------------------------------------------------------------------------------------
%	TITLE SECTION
%----------------------------------------------------------------------------------------

\newcommand{\horrule}[1]{\rule{\linewidth}{#1}} % Create horizontal rule command with 1 argument of height

\title{	
\normalfont \normalsize 
\textsc{Mathematical foundations of computer graphics and vision} \\ [25pt] % Your university, school and/or department name(s)
\horrule{0.5pt} \\[0.4cm] % Thin top horizontal rule
\huge Exercise 5. Rigid Transform Blending and Variational Methods\\ % The assignment title
\horrule{2pt} \\[0.5cm] % Thick bottom horizontal rule
}

\author{Dongho Kang \\ \small 16-948-598} % Your name

\date{\normalsize May 21, 2017} % Today's date or a custom date

\begin{document}

\maketitle % Print the title

%----------------------------------------------------------------------------------------
%	README
%----------------------------------------------------------------------------------------

MATLAB R2016b version was used for coding and testing:

\begin{center}
MathWorks, MATLAB R2016b (9.1.0.441655) \\
64-bit (maci64) 
\end{center}

The \filename{code} directory contains the followings:

\begin{filedescription}
	\item [part1\_1.m] script .m file for exercise part 1.1.
	\item [part1\_2.m] script .m file for exercise part 1.2.
	\item [part2\_1.m] script .m file for exercise part 2.1. 
	\item [part2\_2.m] script .m file for exercise part 2.2. 
	\item [PART I] provided directory for part 1.
	\item [PART II] directory which contains implementation of part 2 and provided files including skeleton code etc.
	\item [img] directory which contains images for testing part 2.2   
	\item [result] result image of part 1 and part 2.
\end{filedescription}

For running each .m script, check dependencies (especially for \filename{part2\_1.m} and \filename{part2\_2.m}) and adjust parameters first. Note that \textbf{these scripts only work properly in MATLAB R2016b environment} and \textbf{have done in Mac OS 10.11.6.} More details are stated in the \textit{Running} section of each parts.

%----------------------------------------------------------------------------------------
%	PROBLEM 1
%----------------------------------------------------------------------------------------

\section{exercise part 1: Understanding and Utilizing Dual Quaternion}

%----------------------------------------------------------------------------------------
%	TASK 1
%----------------------------------------------------------------------------------------
\subsection{Task 1: Thinking about fundamental properties}

\begin{itemize}
	\item How do dual quaternions represent rotations and translations? \\

Unit dual quaternions naturally represent 3D rotation, when the dual part $\vec{q_\epsilon} = 0$ (thus $\hat{\vec{q}} = \vec{q_0} + \epsilon \vec{q_\epsilon} = \vec{q_0}$). Dual quaternion multiplication with unit dual quaternion $\hat{\vec{t}} = 1 + \frac{\epsilon}{2}(t_0 i + t_1 j + t_2 k)$ corresponds to translation by vector $(t_0, t_1, t_2)$ represent 3D translation.\\
	
	\item What is the advantage of representing rigid transformations with dual quaternions for blending? \\
	
The linear combination of dual quaternions does not make artifacts or skin-collapsing effect, thus blending using dual quaternions is fast and more robust than using homogeneous matrix. Moreover, since dual quaternions require only 8 floats per transformation, instead of the 12 required by matrices, they are more memory efficient.\\
	
	\item Briefly explain one fundamental disadvantage of using quaternion based shortest path blending for rotations as compared to linear blend skinning? \\
	
TODO 
Dual quaternions causes "flipping artifacts" which occurs with joint rotations of more than 180 degrees, This is a corollary of the shortest path property: when the other path becomes shorter, the skin changes its shape discontinuously. \\

\end{itemize}


%----------------------------------------------------------------------------------------
%	TASK 2 
%----------------------------------------------------------------------------------------
\subsection{Task 2: Derivations and deeper understanding}

\begin{itemize}
	\item For a dual quaternion $\hat{\vec{q}} = \text{cos} (\hat{\theta} /2) + \hat{\vec{s}} \, \text{sin} (\hat{\theta} /2)$, prove that $\hat{\vec{q}}^t = \text{cos} (t\hat{\theta} /2) + \hat{\vec{s}} \text{sin} (t\hat{\theta} /2)$ \\
	
	Starting with $\hat{\vec{q}}$:
	\begin{align}
		\hat{\vec{q}} &= \text{cos} (\hat{\theta} /2) + \hat{\vec{s}} \,\text{sin} (\hat{\theta} /2)
	\end{align}

	As $\hat{\vec{q}}^t	= \text{exp} \big(t \, \text{log}(\hat{\vec{q}}) \big)$, 
	\begin{align}
		\hat{\vec{q}}^t	&= \text{exp} \big( t \, \text{log}(\hat{\vec{q}}) \big) \\
						&= \text{exp}\Big(t \, \text{log} \big(\text{cos} (\hat{\theta} /2) + \hat{\vec{s}} \,\text{sin} (\hat{\theta} /2) \big) \Big)
	\end{align}

	Plug in $\text{log} \big(\text{cos} (\hat{\theta} /2) + \hat{\vec{s}} \,\text{sin} (\hat{\theta} /2) \big) = \hat{\vec{s}} \frac{\hat{\theta}}{2}$ to equation (1.3):
	\begin{align}
		\hat{\vec{q}}^t = \text{exp}\Big( \frac{t\hat{\theta}}{2} \,\hat{\vec{s}} \Big)
	\end{align}
	
	Let $\hat{\vec{a}} = \frac{t\hat{\theta}}{2} \,\hat{\vec{s}}$. Since $\hat{\theta} = \theta_0 + \epsilon \theta_\epsilon$ and $\hat{\vec{s}} = \vec{s}_0 + \epsilon\vec{s}_\epsilon$.
	\begin{align}
		\hat{\vec{a}} &= \frac{t\hat{\theta}}{2} \hat{\vec{s}} \\
		&= \frac{t}{2}(\theta_0 + \epsilon \theta_\epsilon)(\vec{s}_0 + \epsilon\vec{s}_\epsilon) \\
		&= \frac{t}{2} (\theta_0\vec{s}_0 + \epsilon \theta_0 \vec{s}_\epsilon + \epsilon \theta_\epsilon \vec{s}_0 + \cancelto{0}{\epsilon^2 \theta_\epsilon \vec{s}_\epsilon}) \\
		&= \underbrace{ \Big( \frac{t}{2} \theta_0\vec{s}_0 \Big) }_{\coloneqq \vec{a}_0} 
			+ \epsilon \underbrace{ \Big( \frac{t}{2} \theta_0 \vec{s}_\epsilon + \frac{t}{2} \theta_\epsilon \vec{s}_0 \Big) }_{\coloneqq \vec{a}_\epsilon }
	\end{align}
	
	Since exponential of dual quaternion is given by $e^{\hat{\vec{q}}} = \text{cos} \big( \norm{\hat{\vec{q}}} \big) + \frac{\hat{\vec{q}}}{\norm{\hat{\vec{q}}}} \text{sin} \big( \norm{\hat{\vec{q}}} \big)$, then $e^{\hat{\vec{a}}} = \text{cos} \big( \norm{\hat{\vec{a}}} \big) + \frac{\hat{\vec{a}}}{\norm{\hat{\vec{a}}}} \text{sin} \big( \norm{\hat{\vec{a}}} \big)$. Here, the norm of dual quaternion $\hat{\vec{a}}$ is $\norm{\hat{\vec{a}}} = \norm{\vec{a}_0} + \epsilon \frac{<\vec{a}_0, \vec{a}_\epsilon>}{\norm{\vec{a}_0}}$. Looking into $<\vec{a}_0, \vec{a}_\epsilon>$ and $\norm{\vec{a}_0}$,
	\begin{align}
		<\vec{a}_0, \vec{a}_\epsilon> &= <\frac{t}{2} \theta_0\vec{s}_0, \frac{t}{2} \theta_0 \vec{s}_\epsilon + \frac{t}{2} \theta_\epsilon \vec{s}_0> \\
		&= <\frac{t}{2} \theta_0\vec{s}_0, \frac{t}{2} \theta_0 \vec{s}_\epsilon> + <\frac{t}{2} \theta_0\vec{s}_0, \frac{t}{2} \theta_\epsilon \vec{s}_0> \\[0.5cm] 
		\norm{\vec{a}_0} &= \sqrt{<\vec{a}_0, \vec{a}_0>} \\
		&= \sqrt{<\frac{t}{2} \theta_0 \vec{s}_0, \frac{t}{2} \theta_0 \vec{s}_0>} 
	\end{align}
	
	Note that $<\vec{s}_0, \vec{s}_0> = 1$ and $<\vec{s}_0, \vec{s}_\epsilon> = 0$. Thus equation (1.10) and (1.12) can be expressed as follows: 
	\begin{align}
		<\vec{a}_0, \vec{a}_\epsilon> &= \Big(\frac{t}{2}\Big)^2 \theta_0 \theta_\epsilon \\[0.5cm]
		\norm{\vec{a}_0} &= \sqrt{\Big(\frac{t}{2} \theta_0 \Big)^2} = \frac{t}{2} \theta_0 
	\end{align}
	
	Plug (1.13) and (1.14) into $\norm{\hat{\vec{a}}} = \norm{\vec{a}_0} + \epsilon \frac{<\vec{a}_0, \vec{a}_\epsilon>}{\norm{\vec{a}_0}}$:
	\begin{align}
		\norm{\hat{\vec{a}}} &= \frac{t}{2} \theta_0 + \epsilon \frac{\Big(\frac{t}{2}\Big)^2 \theta_0 \theta_\epsilon}{\frac{t}{2} \theta_0}	\\
		&= \frac{t}{2} \theta_0 + \epsilon \frac{t}{2} \theta_\epsilon \\
		&= \frac{t}{2} (\theta_0 + \epsilon \theta_\epsilon) = \frac{t}{2} \hat{\theta}
	\end{align}
	
	Finally, by (1.4), (1.17) and $e^{\hat{\vec{a}}} = \text{cos} \big( \norm{\hat{\vec{a}}} \big) + \frac{\hat{\vec{a}}}{\norm{\hat{\vec{a}}}} \text{sin} \big( \norm{\hat{\vec{a}}} \big)$, 
	\begin{align}
		\hat{\vec{q}}^t &= e^{\hat{\vec{a}}} \\
		&= \text{cos} \Big( \frac{t}{2} \hat{\theta} \Big) + \frac{\hat{\vec{a}}}{\frac{t}{2}\hat{\theta}} \text{sin} \Big( \frac{t}{2} \hat{\theta} \Big) \\
		&= \text{cos} \Big( \frac{t}{2} \hat{\theta} \Big) + \frac{\cancel{\frac{t}{2} \hat{\theta}} \,\hat{\vec{s}}}{\cancel{\frac{t}{2}\hat{\theta}}} \text{sin} \Big( \frac{t}{2} \hat{\theta} \Big) \\
		&= \text{cos} \Big( \frac{t}{2} \hat{\theta} \Big) + \hat{\vec{s}} \, \text{sin} \Big( \frac{t}{2} \hat{\theta} \Big)
	\end{align}
	
	The proof has been done.
	
	\item 
	
	
\end{itemize}


%----------------------------------------------------------------------------------------
%	PROBLEM 2 
%----------------------------------------------------------------------------------------

\section{exercise part 2: Interactive Segmentation with Graph Cut}

In this section, three different denoising methods were implemented and compared:

\begin{itemize}
	\item Filtering
	\item Heat diffusion
	\item Variational approach
\end{itemize}

%----------------------------------------------------------------------------------------
%	TASK 1
%----------------------------------------------------------------------------------------

\subsection{Task 1: Filtering}

%----------------------------------------------------------------------------------------
%	TASK 2
%----------------------------------------------------------------------------------------

\subsection{Task 2: Heat diffusion}

%----------------------------------------------------------------------------------------
%	TASK 3
%----------------------------------------------------------------------------------------

\subsection{Task 3: Variational approach}


%----------------------------------------------------------------------------------------
%	TASK 4
%----------------------------------------------------------------------------------------

\subsection{Task 4: Comparison}

\begin{itemize}
	\item How can you describe the results? Does any of these methods give better results than the others? 
	\item What are the benefits and drawbacks of each methods?
	\item Can you explain the motivations behind each of the methods? 
\end{itemize}


%----------------------------------------------------------------------------------------
%	REFERENCES
%----------------------------------------------------------------------------------------

\bibliography{reference} 
\bibliographystyle{ieeetr}

\end{document}
